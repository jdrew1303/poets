\documentclass[10pt,final]{article}

\usepackage[final,hidelinks]{hyperref}
\usepackage[final]{listings}
\usepackage{url}
\urlstyle{sf}

\begin{document}
\title{POETS Installation \& Usage}
\author{Tom Hvitved \and Patrick Bahr \and Jesper Andersen}

% hack:
\renewcommand{\today}{\small Department of Computer Science, University of
    Copenhagen\\ Universitetsparken 1, 2100 Copenhagen, Denmark\\
  \{\texttt{hvitved,bahr,jespera}\}\texttt{@diku.dk}\\[2em]August 2013}

\maketitle

\tableofcontents

\section{Preliminaries}
This document describes how to install POETS from scratch and how to
use a running system. In order to install POETS, the following
software is required:
\begin{itemize}
\item The POETS source code. It is available in the Mercurial
  repository at \url{https://bitbucket.org/jespera/poets}.
\item GHC and Cabal. The easiest way to get GHC and Cabal is to
  install the GHC Platform, avaiable from
  \url{http://hackage.haskell.org/platform/}.
\item Thrift. The latest developer version of Thrift can be obtained
  from \url{http://thrift.apache.org/developers/}.
\item At least version 6 of the JDK (tested only on the version from
  Oracle)
\item The Android SDK, available from \url{http://developer.android.com/sdk/}.
\item The Apache Ant tool.
\end{itemize}

\section{The POETS server}\label{sec:poets-server}
Installing the POETS server requires GHC and Thrift.

\subsection{GHC}
First install the latest version of GHC, including Cabal. After
installing GHC, issue the following commands:
\begin{verbatim}
cabal update
cabal install HTTP
ghc-pkg expose ghc-7.4.1
\end{verbatim}
This will install the GHC package HTTP (needed for Thrift), and make
the GHC package available (needed in order to compile reports). Note
that the version number 7.4.1 depends, of course, on the version of
GHC that is installed.

\subsection{Thrift}
Thrift is needed both for the server and the Android clients
communicating with the server. In order to install Thrift, issue the
following commands in the Thrift source library:
\begin{verbatim}
./bootstrap
./configure --without-c_glib --without-csharp --without-java \
 --without-erlang --without-python --without-perl --without-php \
 --without-php_extension --without-ruby --without-haskell \
 --without-go
make
sudo make install
cd lib/hs
cabal install
cd ../java
ant
\end{verbatim}

This installs the \verb|thrift| executable for generating e.g. Haskell
code from Thrift specification files, as well as Thrift bindings for
Haskell and Java. In particular a jar-file
\verb|libthrift-x.x.x-snapshot.jar| should now be present in the
\verb|lib/java/build/| folder (relative to the main folder of Thrift).

\subsection{POETS Thrift library}
The POETS Thrift library is a GHC library generated by Thrift from the
files in the folder \verb|src/thrift| in the POETS source code
folder. It is installed by issuing the following commands in the POETS
repository:
\begin{verbatim}
cd src/haskell/thrift
make
\end{verbatim}

\subsection{The Server}
Finally, the server itself can be installed by issuing the following
commands in the POETS repository:
\begin{verbatim}
cd src/haskell
cabal install
\end{verbatim}
This will install the \verb|poetsserver| executable that is the
server. Note that the first installation may take a rather long time,
since all library dependencies are downloaded and installed as
well.

\section{Android Client}

An Android-based client have been developed for communicating with the
POETS server. The client is meant to be run on Android 3.1 or
upwards. It has been successfully run on the Samsung Galaxy Tab 10.1
running both Honeycomb and Ice Cream Sandwich the Samsung Galaxy S 2
running Ice Cream Sandwich (although the screen on the phone is not
really large enough).

\subsection{POETSIO library}

The Android client makes use of a small utility library called
POETSIO. The source code for POETSIO can be found in
\verb-src/java/poetsio/-. One can package the library in a jar-file by
running the following command from the folder
\verb-src/java/poetsio/-. Before running the command one needs to copy
the Thrift Java library as generated during the installation of Thrift
described above (\verb-libthrift.jar-) to the \verb-lib- subfolder of
the POETSIO folder.
\begin{verbatim}
ant dist
\end{verbatim}
This command will compile the Java classes residing in the \verb-src-
subfolder and package them into a jar-file in
\verb-src/java/poetsio/dist/-.


\subsection{Tablet client}

\paragraph{Building}
To build the Android tablet client follow the below steps:
\begin{enumerate}
\item Copy \verb-libthrift.jar- to the \verb-src/java/tablet/libs/-
  folder.
\item Generate the \verb-poestio.jar- file as described in the
  previous section and copy it to \verb-src/java/tablet/libs/-.
\item Run \verb-ant debug- from the \verb-src/java/tablet/- folder.
\end{enumerate}
The above steps should produce an file in the
\verb-src/java/tablet/bin- folder named \verb|Continuum-debug.apk|.

\paragraph{Installing} To install the tablet client onto an Android
tablet make sure that you have enabled the ability install apps from
non-market sources. (There is probably a toggle somewhere in
\verb-Settings>Security-). To put the app onto the device first
connect the device to your computer via the usb-connection and then
run the following command from the \verb-src/java/tablet/- folder:
\begin{verbatim}
adb -d install -r bin/Continuum-debug.apk
\end{verbatim}

\section{Usage and Configuration}

\subsection{The POETS Server}\label{sec:usage-poets-server}
Building the POETS server as described in
Section~\ref{sec:poets-server} results in the \verb|poetsserver|
executable. The server is started by pointing the executable to a
configuration file, e.g. \verb|poetsserver muERP.config|. The
configuration file determines the debugging level, the location of the
debug log file, and the location of the event log file which is
\emph{the only} persistent state of the running system.

The server can be started on an initially empty event log, which
results in an ``empty system'' that only contains predefined data
definitions, predefined report functions, and predefined contract
functions as described in the tech report~\cite{hvitved11tr}. The
location of these system definitions is:
\begin{verbatim}
src/haskell/defs
\end{verbatim}
and the contents of these files are not supposed to be changed as part
of configuring a running system. Rather the contents of these files
can be thought of as a system prelude that is always present in any
system. Inspecting these files can however be helpful to see
e.g. which event types are present in any system. For example, the
file \verb|src/haskell/defs/data/Event.pce| contains the definition
\begin{verbatim}
Event is abstract.
Event has a DateTime called internalTimeStamp.
\end{verbatim}
which defines the super type of all events that are logged in the
system. Hence all events that are logged in the event log are known to
contain a field called \verb|internalTimeStamp| of the type
\verb|DateTime|.

\subsection{Interface to the POETS Server}
The only interface to a running POETS server is the procedures
described in the Thrift specification file. This file is located at
\verb|src/thrift/poets.thrift|. For a complete list of procedures see
this file.

\subsection{Bootstrapping a Fresh System}
As described in Section~\ref{sec:usage-poets-server} it is possible to start the
POETS server on an initially empty system. In order to get an actually
running system, we must hereafter add data definitions, report
definitions, and contract definitions, as described in the tech
report~\cite[Section~3.4]{hvitved11tr}. In order to ease this process,
rather than adding the definitions manually by invoking the relevant
Thrift procedures described in the previous section, we provide a
quick-and-dirty Java program.

The bootstrap program resides in the folder
\begin{verbatim}
src/java/bootstrap
\end{verbatim}
and it is invoked by issuing the command
\begin{verbatim}
ant boot -Dfile=setup.ini
\end{verbatim}
The ant-script runs that Java bootstrap program that reads which
definitions to add from a simple named configuration file (here named
\verb-setup.ini-) which consists of five parts:
\begin{enumerate}
\item Location of ontology files to add.
\item Location of a custom report prelude.
\item Location of report files.
\item Location of a custom contract prelude.
\item Location of contract files.
\end{enumerate}

The ontology files consists of four files, corresponding to the
division into the root concepts Data, Transaction, Contract, and
Report~\cite{hvitved11tr}.

The custom report prelude is a means to provide common functionality
for report definitions. Since the report engine does not have a proper
module system at the time of writing, the custom prelude is simply
inlined (HACK!) into each report file before it is uploaded. Note that
we cannot solve this problem by moving the custom prelude functions
into the system prelude, as these functions typically depend on
ontology definitions such as Payment that are not necessarily present
in all systems.

The custom contract prelude is similar to that for reports, and the
same inlining hack is used until a proper module system is
implemented.

\bibliographystyle{plain}
\bibliography{refs}

\end{document}