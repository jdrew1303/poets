We have presented an extended and generalised version of the POETS
architecture~\cite{henglein09jlap}, which we have fully
implemented. We have presented domain-specific languages for
specifying the data model, reports, and contracts of a POETS instance,
and we have demonstrated an application of POETS in a small use
case. The use case demonstrates the conciseness of our
approach---Appendix~\ref{app:muerp} contains the complete source
needed for a running system---as well as the domain-orientation
of our specification languages. We believe that non-programmers should
be able to read and understand the data model of
Appendix~\ref{app:muerp-ontology}, to some extent the contract
specifications of Appendix~\ref{app:muerp-contracts-templates}, and to
a lesser extent the reports of Appendix~\ref{app:muerp-reports} (after all,
reports describe computations).

\subsection{Future Work}
\label{sec:future-work}

With our implementation and revision of POETS we have only taken the
first steps towards a software system that can be used in practice. In
order to properly verify our hypothesis that POETS is practically
feasible, we want to conduct a larger use case in a live, industrial
setting. Such use case will both serve as a means of testing the
technical possibilities of POETS, that is whether we can model and
implement more complex scenarios, as well as a means of testing our
hypothesis that the use of domain-specific languages shortens the gap
between requirements and implementation.

\paragraph{Expressivity}
As mentioned above, a larger and more realistic use case is needed in
order to fully evaluate POETS. In particular, we are interested in
investigating whether the data model, the report language, and the
contract language have sufficient expressivity. For instance, a
possible extension of the data model is to introduce finite maps. Such
extension will, for example, simplify the reports from our \muerp use
case that deal with multiple currencies. Moreover, finite maps will
enable a modelling of resources that is closer in structure to that
of Henglein et al.~\cite{henglein09jlap}.

Another possible extension is to allow types as values in the report
language. For instance, the \emph{EntitiesByType} report in
Appendix~\ref{app:muerp-reports-internal} takes a string
representation of a record type, rather than the record type
itself. Hence the function cannot take subtypes into account, that is
if we query the report with input \recordname{A}, then we only get
entities of declared type \recordname{A} and not entities of declared
subtypes of \recordname{A}.

\paragraph{Rules}
A rule engine is a part of our extended architecture
(Figure~\ref{fig:arch}), however it remains to be implemented. The
purpose of the rule engine is to provide rules---written in a separate
domain-specific language---that can constrain the values that are
accepted by the system. For instance, a rule might specify that the
\fieldname{items} list of a \recordname{Delivery} transaction always
be non-empty.

More interestingly, the rule engine will enable values
to be \emph{inferred} from the rules in the engine. For instance, a
set of rules for calculating VAT will enable the field
\fieldname{vatPercentage} of an \recordname{OrderLine} to be inferred
automatically in the context of a \recordname{Sale} record. That is,
based on the information of a sale and the items that are being sold,
the VAT percentage can be calculated automatically for each item
type.

The interface to the rule engine will be very simple: A record value,
as defined in Section~\ref{sec:values}, with zero or more \emph{holes}
is sent to the engine, and the engine will return either
\begin{inparaenum}[(i)]
\item an indication that the record cannot possibly fulfil the rules in the
  engine, or
\item a (partial) substitution that assigns inferred values to (some
  of) the holes of the value as dictated by the rules.
\end{inparaenum}
Hence when we, for example, instantiate the sale of a bicycle in
Section~\ref{sec:bootstrapping}, then we first let the rule engine
infer the VAT percentage before passing the contract meta data to the
contract engine.

\paragraph{Forecasts}
A feature of the contract engine, or more specifically of the reduction
semantics of contract instances, is the possibility to retrieve the
state of a running contract at any given point in time. The state is
essentially the AST of a contract clause, and it describes what is
currently expected in the contract, as well as what is expected in the
future.

Analysing the AST of a contract enables the possibility to do
\emph{forecasts}, for instance to calculate the expected outcome of a
contract or the items needed for delivery within the next
week. Forecasts are, in some sense, dual to reports. Reports derive
data from transactions, that is facts about what has previously
happened. Forecasts, on the other hand, look into the future, in terms
of calculations over running contracts. We have currently implemented
a single forecast, namely a forecast that lists the set of
immediately expected transactions for a given contract. A more
ambitious approach is to devise (yet another) language for writing
forecasts, that is functions that operate on contract ASTs.

\paragraph{Practicality}
In order to make POETS useful in practice, many features are still 
missing. However, we see no inherent difficulties in adding them to
POETS compared to traditional ERP architectures. To mention a few:
\begin{inparaenum}[(i)]
\item security, that is authorisation, users, roles, etc.;
\item module systems for the report language and contract language,
  that is better support for code reuse; and
\item check-pointing of a running system, that is a dump of the memory
  of a running system, so the event log does not have to be replayed
  from scratch when the system is restarted.
\end{inparaenum}

\paragraph{Acknowledgements} We are grateful to Fritz Henglein for
many fruitful discussions and for convincing us of the \emph{POETS
  approach} in the first place. Morten Ib Nielsen and Mikkel
J\o{}nsson Thomsen both contributed to our implementation and design
of POETS, for which we are thankful. Lastly, we thank the participants
of the DIKU course ``POETS Summer of Code'' for valuable input.

%%% Local Variables: 
%%% mode: latex
%%% TeX-master: "tr"
%%% End: 